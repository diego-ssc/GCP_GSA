\documentclass[a4paper]{report}

\usepackage{pgf}
\usepackage[utf8]{inputenc}
\usepackage{verbatim}
\usepackage{titling}
\usepackage{booktabs}
\usepackage{enumitem}
\usepackage{qtree}
\usepackage{amssymb}
\usepackage{amsmath}
\usepackage{times}
\usepackage{dsfont}
\usepackage{titling}
\usepackage{cite}
\usepackage[spanish]{babel}
\usepackage{svg}
\usepackage{titlesec}

\pretitle{\begin{center}\linespread{1}}
  \posttitle{\end{center}\vspace{0.14cm}}
\preauthor{\begin{center}\Large}
  \postauthor{\end{center}}

\setlength{\droptitle}{-10em}
\title {\textbf {\Large{Problema de Coloraci\'on de Gr\'aficas}}\protect\\
  \large{\textbf{Algoritmo de B\'usqueda Gravitacional}}\protect\\ \vspace{0.4cm}
  \normalsize{\textbf{Seminario de Ciencias de la Computaci\'on B}} \protect\\ \vspace{0.2cm}
  \normalsize{Canek Pel\'aez Vald\'es} \protect\\ \vspace{0.4cm}
  \normalsize{Universidad Nacional Aut\'onoma de M\'exico}}
\date{}
\author{\normalsize Sánchez Correa Diego Sebastián}

\makeatletter
\renewcommand{\@makechapterhead}[1]{%
  \vspace*{50 pt}%
  {\setlength{\parindent}{0pt} \raggedright \normalfont
    \bfseries\Huge\thechapter.\ #1
    \par\nobreak\vspace{40 pt}}}
\makeatother

\begin{document}
\allowdisplaybreaks
\maketitle

\chapter{Introducci\'on}
El problema de coloraci\'on de gr\'aficas, espec\'ificamente la modalidad que involucra
la b\'usqueda del n\'umero crom\'atico de la gr\'afica, es uno de los problemas NP-duros
que tienen una gran aplicaci\'on en la resoluci\'on de problemas del mundo real.

Una de estas es la optimizaci\'on del uso de los registros disponibles en un procesador
para la ejecuci\'on de un algoritmo; problema para el que es encontrado r\'apidamente
una aproximaci\'on por los compiladores pero del cual, sin embargo, no se puede garantizar
una soluci\'on \'optima, dada la naturaleza del problema.

\section{El problema}
Una gr\'afica se define como una par ordenado compuesto de un conjunto de v\'ertices $V$ y
un conjunto de aristas $E$, es decir, una gr\'afica es representada como sigue
\[G =  (V, E)\]

donde

\begin{align*}
  V &= \{v_1, ..., v_n\}\\
  E &\subseteq V \times V
\end{align*}

El problema es planteado sobre una gr\'afica no dirigida, es decir, donde se tiene que
\[(v, w) \in E \Rightarrow (w, v) \in E\]

El vecindaro de un v\'ertice se define como todos los v\'ertices para los cuales existe
una arista que los une, es decir

\[N(v) = \{w \in V \ | \ (v, w) \in E\}\]

El problema de coloraci\'on de gr\'aficas pregunta cu\'al es la mejor asignaci\'on de
colores para cada uno de los v\'ertices contenidos en la gr\'afica que minimice el
n\'umero de colores usados, cumpliendo en todo caso que dos colores adyacentes no
compartan el mismo color.

El n\'umero crom\'atico define, dada una gr\'afica no dirigida, el n\'umero de colores
m\'inimo que debe tener para colorearla por completo.
Dadas las condiciones del problema, es posible obtener m\'ultiples soluciones, ninguna
de ellas pudiendo llamarse la m\'as \'optima (derivado de la naturaleza no determinista
del problema).

\section{La heur\'istica}
La heur\'istica de b\'usqueda gravitacional (GSA) propone un modelo de b\'usqueda que
sigue un patr\'on de enjambre, es decir, no existe un solo agente que realice la
b\'usqueda; esta se divide entre diversos entes que permiten una mayor exploraci\'on
y, eventualmente, una mayor explotaci\'on.

\begin{center}
  \textit{La inteligencia de enjambre es un modelo donde el comportamiento colectivo
    emergente es el resultado de un proceso de organizaci\'on personal, donde los
    agentes est\'an envueltos, a trav\'es sus acciones repetidas e interacci\'on,
    con su ambiente en evoluci\'on. ~\cite{israel}}
\end{center}

Modelos de este tipo son bastante populares; entre ellos se encuentran PSO (Eberhart
and Kennedy, 1995) y ACS (M. Dorigo and V. Maniezzo, 2008).

En esta heur\'istica se plantea el uso de entes (soluciones) que
se afectan m\'utuamente a trav\'es de fen\'omenos f\'isicos, siguiendo las leyes
de Newton.

\begin{center}
  \textit{En el algoritmo propuesto los agentes que buscan son una colecci\'on de masas
    que interact\'uan entre s\'i bas\'andose en la gravedad Newtoniana y las leyes
    del movimiento. ~\cite{gsa}}
\end{center}

A pesar de ser esta una de las primeras concepciones de los fen\'omenos como soluciones
a problemas combinatorios, se usar\'a una variaci\'on que lo ha tenido como
inspiraci\'on y es descrito en ~\cite{israel}.

Se define un conjunto de agentes

\[B = \{b_1, b_2, ... b_N\}\]

correspondientes a cada nodo de la gr\'afica. Cada agente navegar\'a sobre un
mundo t\'orico \footnote{Un toro es \textit{la colecci\'on de todos los puntos
    $(x, y) \in \mathbb{R}^2$ bajo la relaci\'on de equivalencia $(x, y) \sim
    (a, b)$ cuando $x - a, y-b \in \mathbb{Z}$}~\cite{toroidal}} de acuerdo
a un vector $\vec{v_i}$. En cada momento sabemos la posici\'on de cada agente
$p_i(t) = (x_i, y_i)$ donde $x_i$ y $y_i$ son coordenadas cartesianas en el
espacio. Cuando $t = 0$, tenemos la posici\'on inicial de los agentes $p_i(0) = (x_{0i}, y_{0i})$.

Suponemos que queremos colorear la gr\'afica con $K$ colores, denotando
como $C = \{1, 2, ..., K\}$ al conjunto de colores, donde $K$ no debe ser menor
que el n\'umero crom\'atico asociado a la gr\'afica para que el algoritmo
converja. Asignamos a estos colores, $K$ puntos en el espacio, los colores
objetivo $CG = \{g_1, ..., g_K\}$ dotados de una atracci\'on gravitacional
resultando en un componente de velocidad $\vec{v_{gc}}$ afectando a los agentes.
La fuerza de atracci\'on disminuir\'a con la distancia, pero afectar\'a a todos
los agentes en el espacio.

~\cite{israel} define el sistema como una tupla:

\[F = (B, CG, \{\vec{v_i}\}, K, \{\vec{a_{i, k}}\}, R)\]

donde:

\begin{itemize}
\item $B$ es el conjunto de agentes
\item $\{\vec{v_i}\}$ es el conjunto de vectores en el instante $t$
\item $K$ es el n\'umero crom\'atico hipot\'etico \footnote{Este se definir\'a
    como el n\'umero crom\'atico para gr\'aficas creadas artificialmente y
    como un n\'umero aleatorio (lo suficientemente grande) para gr\'aficas, de la
    misma manera, aleatorias.}
\item $\{\vec{a_{i, k}}\}$ es el conjunto de fuerzas de atracci\'on de los colores
  objetivo ejercidas en los agentes.
\item $R$ denota las fuerzas de repulsi\'on en el vecindario de los colores objetivo.
\end{itemize}

\subsection{Velocidad de un agente}
La velocidad de una agente est\'a definida de la siguiente manera:

\begin{align*}
  \vec{v_i}(y+1) =
  \begin{cases}
    0 & c_i \in C \ \& \ (\lambda_i = 1)\\
    d \cdot \vec{a_{i, k^*}} & c_i \not \in C\\
    v_r \cdot (p_r - p_i) & (c_i \in C) \ \& \ (\lambda_i = 0)
  \end{cases}
\end{align*}

donde

\begin{itemize}
\item $d$ es la distancia de la posici\'on $p_i$ del agente a la posici\'on del color
  objetivo m\'as cercano $g_{k^*}$
\item $\vec{a_{i, k^*}}$ representa la fuerza de atracci\'on que el color objetivo
  m\'as cercano ejerce sobre el agente
\item $v_r$ es la magnitud de un vector aleatorio moviendo al agente hacia una
  direcci\'on aleatoria $p_r$ cuando este es expulsado de un color objetivo.
\item $\lambda_i$ es un par\'ametro que representa el efecto del grado de
  confort del agente. Cuando un agente $b_i$ alcanza el radio de influencia de
  un color objetivo en un instante $t$, su velocidad se torna 0.
\end{itemize}

\subsection{Din\'amica dentro de un color objetivo}

La interacci\'on de los agentes y los colores objetivo se define a trav\'es de un
radio de influencia generado por los colores. Y cuando el agente est\'e dentro
de este, dejar\'a de moverse y el nodo correspondiente en la gr\'afica ser\'a
asignado a este color. Se denota al conjunto de agentes cuya posici\'on en el
espacio es la regi\'on que se encuentra los suficientemente cerca al vecindario
de un color como

\[N(g_k) = \{b_i \ \text{ t.q. } ||p_i - g_k|| < \textit{r}\}\]
\begin{center}
  \tiny{\textit{Donde $r$ representa el radio de influencia del color.}}
\end{center}

A pesar de que dentro del radio de influencia del color no hay m\'as atracci\'on
gravitacional, puede estar presente cierto grado de repulsi\'on entre agentes
que est\'an conectados a trav\'es de una arista en la gr\'afica $G$. Esta
repulsi\'on solo es efectiva para los agentes dentro del vecindario del mismo
color objetivo. Este efecto se modela definiendo una funci\'on que tiene
un valor de $1$ si un par de agentes tienen una arista en com\'un, y 0, en
otro caso. Las fuerzas de repulsi\'on que experimenta el agente $b_i$ de los
agentes en el color objetivo $g_k$ se definen como sigue:

\[R(b_i, g_k) = \sum_{N(g_k)} \textit{repulsi\'on} \ (b_i, b_j)\]

\subsection{Confort}

En cada paso del proceso iterativo en el que un agente permanece en un color
objetivo sin ser perturbado, su confort aumenta (hasta llegar a un m\'aximo
definido). Cuando otro agente $b_i$, fuera del color objetivo $g_{k^*}$ intenta
entrar al vecindario de ese color objetivo, la fuerza de repulsi\'on $R(b_i, g_{k^*})$
es evaluada. Si la repulsi\'on es mayor que cero entonces el agente entrante
desafiar\'a la estabilidad del vecindario y al menos un agente tendr\'a que
abandonarlo (el cual puede ser \'el mismo). Si los valores de confort
de los agentes desafiados tienen valores mayores a cero, entonces su confort
disminuir\'a en una unidad. Si su confort llega a cero, entonces el agente
es expulsado del color objetivo a una posici\'on $p_r$ aleatoria en el espacio
con velocidad $v_r$.

\subsection{Funci\'on de Costo}
La funci\'on de costo definida en la configuraci\'on espacial del sistema global es:
\[f(B, CG) = |\{b_i  \ \textit{ t.q. } \ c_i \in C \ \& \ R(b_i, g_{ci}) = 0\}|.\]

Esta funci\'on de costo es el n\'umero de nodos de la gr\'afica que tienen un color
asignado y sin ning\'un conflicto dentro de un color objetivo. Los agentes fuera
del vecindario de un color objetivo no pueden ser evaluados, para que estos sean
parte de la soluci\'on al problema.

Se destacan a la dimensi\'on del mundo y al radio de influencia de los colores
objetivo como factores importantes para la determinaci\'on de la velocidad de
convergencia del algoritmo. Si el mundo es grande y el radio peque\~no el
algoritmo converger\'a lentamente, mon\'otonamente; si el mundo es reducido
y el radio es grande el algoritmo ser\'a m\'as r\'apido pero la convergencia
ser\'a inestable porque el algoritmo caer\'a en m\'inimos locales y necesitar\'a
un aumento en la energ\'ia transitoria para salir de ellos. La explicaci\'on de
este comportamiento es que el mundo no est\'a normalizado y la magnitud del vector
velocidad puede ser m\'as grande que el radio de influencia del color objetivo
y puede cruzar un color sin caer en \'el.

Cuando todos los agentes se detienen, tenemos que $f(B, CG) = n$ y la asignaci\'on
de $K$ colores a la gr\'afica $G$ ha sido concluida exitosamente.

\chapter{Dise\~no}
A pesar de que, inicialmente, se sigui\'o como gu\'ia el algoritmo descrito en
~\cite{gsa} (y la versi\'on discreta del mismo, en ~\cite{bgsa}), opt\'e por
implementar la versi\'on dada en ~\cite{israel}; esto porque a pesar de
no basarse en la descripci\'on original, donde el autor conceb\'ia a las soluciones
como agentes que eran modificados y atra\'idos a otras soluciones, planteaba una
algoritmo similar (siguiendo la misma \textit{meta idea}) pero
ya estaba completamente adaptado a los t\'erminos que el problema de coloraci\'on
de gr\'aficas requer\'ia.

\section{La Gr\'afica}

En este nuevo algoritmo se conceb\'ia a la gr\'afica como un mundo t\'orico donde
cada uno de los agentes ya no correspond\'ian a una soluci\'on del problema, sino
a los nodos de la gr\'afica.

\section{Analizador Sint\'actico}
Siguiendo las gu\'ias de dise\~no de programas similares que buscan encontrar el
n\'umero crom\'atico de gr\'aficas aleatorias, encontr\'e que existe un est\'andar
de representaci\'on de gr\'aficas.

\textit{DIMACS} consiste en un formato para gr\'aficas no dirigidas e involucra
las siguientes reglas de sintaxis.

\begin{itemize}
\item Un archivo de entrada contiene toda la informaci\'on de una gr\'afica no
  dirigida.
\item Los nodos est\'an numerados de $1$ a $n$ (siendo este el n\'umero total
  de nodos).
\item Se asume que los archivos de entrada se encuentran en un estado l\'ogicamente
  correcto y consistentes: los identificadores de los nodos son v\'alidos, los nodos
  son \'unicos y exactamente $m$ aristas son definidas.
\item Los comentarios pueden aparecer en cualquier parte del archivo. Cada comentario
  comienza con el car\'acter \textit{c}.
\item Debe haber una \textit{l\'inea del problema} por archivo. Esta debe aparecer
  antes de cualquier nodo o cualquier l\'inea descriptiva. Sigue el siguiente formato:

  \begin{center}
    \texttt{p FORMAT NODES EDGES}
  \end{center}

  Donde \texttt{p} denota que se trata de la l\'inea del problema. El campo \texttt{FORMAT}
  deber\'ia contener la palabra \textit{edge}\footnote{Solo se trata de una convenci\'on
    que mantiene consistencia con desaf\'ios previos relacionados al formato.}. El campo
  \texttt{NODES} contiene un entero que define \textit{n} (el n\'umero de nodos). Mientras
  que el campo \texttt{EDGES} contiene un entero que define \textit{m} (el n\'umero de aristas).

\item Habr\'a un descriptor de arista por cada arista de la gr\'afica, cada uno con el siguiente
  formato:

  \begin{center}
    \texttt{e u v}
  \end{center}
  Representa la arista (u, v), esta aperecer\'a solo una vez y no deber\'a ser repetida como
  (v, u).
\end{itemize}

\section{El problema}

Se plante\'o el problema como uan abstracci\'on de una gr\'afica que posee un significado geom\'etrico;
es decir, esta vivir\'a en un mundo cartesiano (t\'orico, espec\'ificamente).

La gr\'afica posee un arreglo de nodos y una matriz de adyacencias. Ambos poseyendo la ventaja de
realizar b\'usquedas en tiempo constante (aprovechando el hecho de que no ser\'an de tama\~no
variable durante la ejecuci\'on del algoritmo); el primero, abstrayendo los elementos como nodos
que tienen coordenadas e identificador asociados; mientras que el segundo pretende acceder de
manera r\'apida a las conexiones de la gr\'afica a partir de los indentificadores y donde dos
v\'ertices $v_i$ y $v_j$ estar\'an conectados si hay un $1$ en la coordenada $(i, j)$ de la matriz
(evitando abstracciones en estructuras, como en el primer caso).

\section{Algoritmo de B\'usqueda Gravitacional}

La heur\'istica se ejecuta a partir de un solo ciclo que verifica, en cada paso iterativo,
si la funci\'on de costo (eval\'ua cu\'antos nodos tienen un color asociado) devuelve
el n\'umero de nodos total. En adici\'on, contemplando el caso en el cual el n\'umero
crom\'atico hipot\'etico sea menor al real, se cuenta con un n\'umero de iteraciones
m\'aximas que previenen una ejecuci\'on sin fin. Cabe destacar que si el algoritmo termina
de esta \'ultima forma, la heur\'istica no habr\'a encontrado una soluci\'on para el
problema.

\chapter{Implementaci\'on}
La implementaci\'on sigui\'o un modelo orientado a objetos (sin la inclusi\'on de herencia/
polimorfismo) que pretende organizar, modularizar y facilitar el desarrollo y el mantenimiento
del c\'odigo en el lenguaje de programaci\'on C.

Cada una de los archivos representa una clase y contiene funciones asociadas a la misma,
teniendo como par\'ametro al objeto en cuesti\'on.

Las \texttt{structs} (estructuras) se definen dentro de los archivos de c\'odigo y no dentro
de los encabezados. Esto con el fin de restringir el acceso a los elementos privados a los
\textit{getters} y \textit{setters}, as\'i como la creaci\'on de objetos del mismo tipo,
designando el trabajo a los constructores.

\section{Problemas}
La administraci\'on de memoria en un lenguaje de programaci\'on como C conlleva un orden m\'as
severo que permite una menor abstracci\'on en su uso para gozar de una mayor eficiencia.

Fugas de memoria estuvieron presentes, sobre todo en la devoluci\'on de apuntadores desde las
funciones; se torna complicado administrar su uso y liberaci\'on al contar con una cantidad
considerable de funciones de este tipo.

\subsection{Generador de N\'umeros Aleatorios}
En ciertos casos, en la generaci\'on de las coordenadas los colores de la gr\'afica
se contaba con una repetici\'on en las coordenadas $x$ y $y$, algo singular, ya que
hubo una invocaci\'on de la funci\'on $drand48\_r$\footnote{Se trata de una funci\'on
  generadora de n\'umeros aleatorios que es compatible con la paralelizaci\'on del sistema;
  a pesar de que esta no se haya implementado en el sistema.} para cada una de ellas.
Se revisar\'a con m\'as detalle y se espera corregir en un futuro pr\'oximo.

\section{Par\'ametros}
El programa se ejecuta en la l\'inea de comandos, recibiendo los par\'ametros de la misma
como par\'ametros del programa y a la gr\'afica en un archivo con formato
DIMACS en un archivo de texto.

\begin{center}
  \texttt{./GCP\_GSA -f [archivo de entrada] [opciones]}\\
  \tiny{Comando de ejecuci\'on del algoritmo}
\end{center}

Donde las opciones incluyen:
\begin{itemize}
\item \texttt{-s} \ \ la semilla
\item \texttt{-n} \ \ el n\'umero crom\'atico hipot\'etico
\item \texttt{-d} \ \ la dimensi\'on de la gr\'afica\footnote{Con esto me refiero al
    tama\~no de una lado de la gr\'afica, es decir, del mundo t\'orico sobre el que
    se ejecutar\'a la heur\'istica, dado que se ha decidido que esta sea cuadrada.}
\item \texttt{-i} \ \ el n\'umero de iteraciones m\'aximo (en caso de no satisfacer
  la funci\'on de costo)
\item \texttt{-c} \ \ el valor de confort m\'aximo
\item \texttt{-v} \ \ la ejecuci\'on verbosa del algoritmo\footnote{Mostrar\'a las coordenadas
    finales de los agentes y colores.}
  \item \texttt{-r} \ \ el radio de los colores objetivo
\end{itemize}

\chapter{Experimentaci\'on}

La experimentaci\'on se bas\'o en el uso de gr\'aficas predefinidas para el problema
de coloraci\'on de gr\'aficas. Espec\'ificamente, se us\'o una gr\'afica de 300 nodos
y 21695 aristas para la cual se tiene un n\'umero crom\'atico de 28. El algoritmo fue
capaz de obtener resultados decentes en cuesti\'on de unos segundos. Para este caso,
se corri\'o con una hip\'otesis de 80, 100,000 iteraciones y un mundo t\'orico de
dimensi\'on 100, con un radio de 50 para los colores objetivo. El mejor resultado
obtenido fue de 36 (todos rondaron las cuatro decenas).

\chapter{Conclusiones}

A pesar de que la experimentaci\'on no fue exhaustiva, los resultados obtenidos parecen
establecer una base s\'olida para futuros experimentos con una gama m\'as amplia de
gr\'aficas no generadas artificialmente y con una implementaci\'on de concurrencia que
permita una generaci\'on m\'as r\'apida y eficiente de resultados.

\bibliography{ref}{}
\bibliographystyle{plain}
\end{document}